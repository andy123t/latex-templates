% LaTeX LR-notes template
% !TEX program = pdflatex

\documentclass[11pt]{article}

%\usepackage{ctex}
\usepackage{amsmath,amssymb}
\usepackage{graphicx,fancybox,multirow}
\usepackage{mathrsfs}
\usepackage{color,xcolor}
\usepackage{framed}
\usepackage{caption}
\usepackage{hyperref}
\hypersetup{colorlinks=true,linkcolor=blue,
filecolor=blue,citecolor=black,urlcolor=cyan}

\usepackage{enumitem}
%\setlist{leftmargin=*} % noitemsep
%\renewcommand{\labelenumi}{[\arabic{enumi}]}
%\renewcommand{\labelenumii}{\alph{enumii}.}

% layout setting
\usepackage{geometry}
\geometry{left=3.0cm,right=3.0cm,top=2.5cm,bottom=2.5cm}
%\geometry{left=1.25in,right=1.25in,top=1in,bottom=1in}
\setlength{\headheight}{15pt}
\setlength{\headsep}{16pt}

%\renewcommand{\baselinestretch}{1.1}

%--- define title page ---
\makeatletter
\def\@maketitle{%
  \newpage
  \null
  \vspace{-1.5em}%
  \begin{center}%
  \let \footnote \thanks
    {\LARGE \bfseries \@title \par}%
    \vskip 1.0em%
    {\large
      \lineskip .5em%
      \begin{tabular}[t]{c}%
        \@author
      \end{tabular}\par}%
    \vskip 0.5em%
    {\large \@date}%
  \end{center}%
  \par
  \vskip 1.5em}
\makeatother

%--- define color ---
\definecolor{winered}{rgb}{0.6,0,0}
\definecolor{deepblue}{rgb}{0,0,0.8}
\definecolor{deepred}{rgb}{0.8,0,0}

%--- counter of paper ---
\makeatletter
\newcommand{\twodigit}[1]{\two@digits{#1}}
\makeatother
\newcounter{papernum}
\newenvironment{paper}[1][\unskip]{
\stepcounter{papernum}
\bigskip \hrule \vspace{3ex}
\noindent {\bfseries\large Paper~\twodigit{\arabic{papernum}}}\quad #1 \par\vspace{8pt}}{\par\vspace{2pt}}
%--- command of paper ---
\newcommand{\titled}[1]{\noindent{\large\color{deepblue}\sc Title:} {\bfseries #1} \par\vspace{6pt}} %\hangpara{2.0em}{1}
\newcommand{\info}[2]{\noindent{\large\color{deepblue}\sc Info:} #1, {\bfseries ~#2} \par\vspace{6pt}}
\renewcommand{\abstract}[1]{\noindent{\large\color{deepblue}\sc Abstract:} #1 \par\vspace{6pt}}
\newcommand{\keywords}[1]{\noindent{\large\color{deepblue}\sc Keywords:} #1 \par\vspace{6pt}}
\newcommand{\summary}[1]{\noindent{\large\color{deepred}\sc Summary:} #1 \par\vspace{6pt}}
\newcommand{\problem}[1]{\noindent{\large\color{deepblue}\sc Problem:} #1 \par\vspace{6pt}}
\newcommand{\method}[1]{\noindent{\large\color{deepblue}\sc Method:} #1 \par\vspace{6pt}}
\newcommand{\intro}[1]{\noindent{\large\color{deepblue}\sc Intro:} #1 \par\vspace{6pt}}
\newcommand{\conclusion}[1]{\noindent{\large\color{deepblue}\sc Conclusion:} #1 \par\vspace{6pt}}
\newcommand{\note}[1]{\noindent{\large\color{deepred}\sc Note:} #1 \par\vspace{6pt}}
\newcommand{\refs}[1]{\noindent{\large\color{deepblue}\sc References:} #1 \par\vspace{6pt}}

%\numberwithin{equation}{section}

% define new command
\newcommand{\red}[1]{{\color{red}#1}}
\newcommand{\blue}[1]{{\color{blue}#1}}
\newcommand{\winered}[1]{\textcolor{winered}{#1}}
\newcommand{\textcode}[1]{\textcolor{winered}{\bfseries\texttt{#1}}}

% path of figures
\graphicspath{{./figures/}}


% Information of Document
\title{Literature Review Notes}
\author{Author XX}
\date{Nov. 8, 2022}


\begin{document}

\maketitle


% main body of document

\section*{Research Topic}

% Paper 1

\begin{paper}

\titled{Title of paper}

\info{Author}{Journal (1998)}

\abstract{Abstract here.}

\keywords{Keyword 1, keyword 2.}

\summary{Summary here. Note: When writing a summary, remember that you need to describe it in your own words through your own thinking. Ctrl + C is taboo.}

\problem{The objective of the paper and the problem studied in the paper.}

\method{The method for solving the model problem.}

\intro{Lorem ipsum dolor sit amet, consectetur adipiscing elit.
Proin eu tempor velit. Fusce accumsan ultrices fringilla. Praesent
sed odio mi. Mauris non ligula turpis. Duis posuere lacus nec diam
interdum dictum suscipit magna molestie. Vestibulum nulla egestas aliquam.
}

\conclusion{Conclusion here.}

\note{Note here.}

\refs{List the literature with high relevance.
\begin{enumerate}[label={[\arabic*]}]
\setlength{\itemsep}{0pt}
  \item Literature 1
  \item Literature 2
\end{enumerate}
}

\end{paper}

% Paper 2

\begin{paper}

\titled{Title of paper}

\info{Author}{Journal (1998)}

\abstract{Abstract here.}

\keywords{Keyword 1, keyword 2.}

\summary{Summary here. Note: When writing a summary, remember that you need to describe it in your own words through your own thinking. Ctrl + C is taboo.}

\problem{The objective of the paper and the problem studied in the paper.}

\method{The method for solving the model problem.}

\intro{Lorem ipsum dolor sit amet, consectetur adipiscing elit.
Proin eu tempor velit. Fusce accumsan ultrices fringilla. Praesent
sed odio mi. Mauris non ligula turpis. Duis posuere lacus nec diam
interdum dictum suscipit magna molestie. Vestibulum nulla egestas aliquam.
}

\conclusion{Conclusion here.}

\note{Note here.}

\end{paper}




\end{document}

