% LaTeX LR-notes template
% !TEX program = xelatex

\documentclass[11pt]{article}

\usepackage{ctex}
\usepackage{amsmath,amssymb}
\usepackage{graphicx,fancybox,multirow}
\usepackage{mathrsfs}
\usepackage{color,xcolor}
\usepackage{framed}
\usepackage{caption}
\usepackage{hyperref}
\hypersetup{colorlinks=true,linkcolor=blue,
filecolor=blue,citecolor=black,urlcolor=cyan}
\definecolor{winered}{rgb}{0.6,0,0}

\usepackage{enumitem}
%\setlist{leftmargin=*} % noitemsep
%\renewcommand{\labelenumi}{[\arabic{enumi}]}
%\renewcommand{\labelenumii}{\alph{enumii}.}

% layout setting
\usepackage{geometry}
\geometry{left=3.0cm,right=3.0cm,top=2.5cm,bottom=2.5cm}
%\geometry{left=1.25in,right=1.25in,top=1in,bottom=1in}
\setlength{\headheight}{15pt}
\setlength{\headsep}{16pt}

%--- define title page ---
\makeatletter
\def\@maketitle{%
  \newpage
  \null
  \vspace{-1.5em}%
  \begin{center}%
  \let \footnote \thanks
    {\LARGE \bfseries \@title \par}%
    \vskip 1.0em%
    {\large
      \lineskip .5em%
      \begin{tabular}[t]{c}%
        \@author
      \end{tabular}\par}%
    \vskip 0.5em%
    {\large \@date}%
  \end{center}%
  \par
  \vskip 1.5em}
\makeatother

%--- counter of paper ---
\makeatletter
\newcommand{\twodigit}[1]{\two@digits{#1}}
\makeatother
\newcounter{papernum}
\newenvironment{paper}[1][\unskip]{
\stepcounter{papernum}
\bigskip \hrule \vspace{3ex}
\noindent {\bfseries\large 论文~\twodigit{\arabic{papernum}}}\quad #1 \par\vspace{8pt}}{\par\vspace{2pt}}
%--- command of paper ---
\newcommand{\titled}[1]{\noindent{\large\color{blue} 题目:} #1 \par\vspace{6pt}} %\hangpara{2.0em}{1}
\newcommand{\info}[1]{\noindent{\large\color{blue} 信息:} #1 \par\vspace{6pt}}
\newcommand{\abst}[1]{\noindent{\large\color{blue} 摘要:} #1 \par\vspace{6pt}}
\newcommand{\keywords}[1]{\noindent{\large\color{blue} 关键词:} #1 \par\vspace{6pt}}
\newcommand{\summary}[1]{\noindent{\large\color{red} 总结:} #1 \par\vspace{6pt}}
\newcommand{\problem}[1]{\noindent{\large\color{blue} 问题:} #1 \par\vspace{6pt}}
\newcommand{\method}[1]{\noindent{\large\color{blue} 方法:} #1 \par\vspace{6pt}}
\newcommand{\intro}[1]{\noindent{\large\color{blue} 内容:} #1 \par\vspace{6pt}}
\newcommand{\conclusion}[1]{\noindent{\large\color{blue} 结论:} #1 \par\vspace{6pt}}
\newcommand{\note}[1]{\noindent{\large\color{red} 注:} #1 \par\vspace{6pt}}
\newcommand{\refs}[1]{\noindent{\large\color{blue} 文献:} #1 \par\vspace{6pt}}

%\numberwithin{equation}{section}

% define new command
\newcommand{\red}[1]{{\color{red}#1}}
\newcommand{\blue}[1]{{\color{blue}#1}}
\newcommand{\winered}[1]{\textcolor{winered}{#1}}
\newcommand{\textcode}[1]{\textcolor{winered}{\bfseries\texttt{#1}}}

% path of figures
\graphicspath{{./figures/}}


% Information of Document
\title{文献综述笔记}
\author{某某某}
\date{2022.11.08}


\begin{document}

\maketitle


% main body of document

\subsection*{XXX}

% Paper 2

\begin{paper}

\titled{Title of paper}

\info{Author, Journal, 1998}

\abst{Abstract here.}

\keywords{Keyword 1, keyword 2.}

\summary{Summary here. Note: When writing a summary, remember that you need to describe it in your own words through your own thinking. Ctrl + C is taboo.}

\method{The method for solving the model problem.}

\intro{Lorem ipsum dolor sit amet, consectetur adipiscing elit.
Proin eu tempor velit. Fusce accumsan ultrices fringilla. Praesent
sed odio mi. Mauris non ligula turpis. Duis posuere lacus nec diam
interdum dictum suscipit magna molestie. Vestibulum nulla egestas aliquam.
}

\conclusion{Conclusion here.}

\note{Note here.}

\refs{
\begin{enumerate}[label={[\arabic*]}]
  \item Literature 1
  \item Literature 2
\end{enumerate}

}

\end{paper}

% Paper 2

\begin{paper}

\titled{Title of paper}

\info{Author, Journal, 1998}

\abst{Abstract here.}

\keywords{Keyword 1, keyword 2.}

\summary{Summary here. Note: When writing a summary, remember that you need to describe it in your own words through your own thinking. Ctrl + C is taboo.}

\method{The method for solving the model problem.}

\intro{Lorem ipsum dolor sit amet, consectetur adipiscing elit.
Proin eu tempor velit. Fusce accumsan ultrices fringilla. Praesent
sed odio mi. Mauris non ligula turpis. Duis posuere lacus nec diam
interdum dictum suscipit magna molestie. Vestibulum nulla egestas aliquam.
}

\conclusion{Conclusion here.}

\note{Note here.}

\end{paper}





\end{document}
